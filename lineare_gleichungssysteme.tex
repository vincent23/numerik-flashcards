\cardfrontfoot{Lineare Gleichungssysteme}

\begin{flashcard}[Algorithmus]{Rückwärtssubstitution}
\begin{algorithm}[H]
	\KwIn{$R, y$}
	\KwOut{$x$}
	$x_n = y_n/r_{nn};$\\
	\For{$i = n-1, \ldots, 1$}{
		$sum = r_{in}x_n;$\\
		\For{$j = i+1; \ldots, n-1$}{
			$sum = sum + r_{ij}x_j;$
			}
		$x_i = (y_i - sum) / r_{ii};$
		}
\end{algorithm}
\end{flashcard}

\begin{flashcard}[Algorithmus]{LR-Zerlegung}
\begin{enumerate}
	\item Bestimme Matrizen $P$, $L$ und $R$ mit $PA = LR$ durch Gauß-Elimination.
	\item Löse $Ly = Pb$ (Vorwärtssubstitution).
	\item Löse $Rx = y$ (Rückwärtssubstitution).
\end{enumerate}
\end{flashcard}

\begin{flashcard}[Aufwand]{Vorwärts-, Rückwärtssubstitution}
$$\sum_{i=1}^{n-1}i = n \frac{n-1}{2} \approx \frac{n^2}{2}$$
\end{flashcard}

\begin{flashcard}[Aufwand]{LR-Zerlegung}
$$\sum_{j=1}^{n-1}j^2 = \frac{(n-1)n(2n-1)}{6} \approx \frac{n^3}{3}$$
\end{flashcard}

\begin{flashcard}{Spaltenpivotwahl LR-Zerlegung}
$$|a_{jk}^{(k-1)}| = \max_{k \leq i \leq n} |a_{ik}^{(k-1)}|$$
Dies führt zu:\\
$l_{ij} \leq 1$ für alle $i, j$
\end{flashcard}

\begin{flashcard}[Definition]{zugehörige Matrixnorm}
Sei ||.|| eine beliebige Norm auf $\mathbb{R}^n$. Dann definieren wir die \textbf{zugehörige Matrixnorm} auf dem Raum der quadratischen $(n \times n)$-Matrizen durch
$$||A|| := \sup_{x \neq 0} \frac{||Ax||}{||x||} \text{für} A \in \mathbb{R}^{n \times n} \text{.}$$
\end{flashcard}

\begin{flashcard}{Submultiplikativität der Matrixnorm}
$$||I|| = 1,$$\\
$$||A \cdot B|| \leq ||A|| \cdot ||B||.$$
\end{flashcard}

\begin{flashcard}[Satz]{Matrixnormen}
Sei $A$ eine quadratische $(n \times n)$-Matrix. Es gilt:
\begin{enumerate}[(i)]
	\item $\displaystyle ||A||_1 = \max_{j=1,\ldots,n} \sum_{i=1}^n |a_{ij}| \ \text{(Spaltensummennorm)}$
	\item $\displaystyle ||A||_2 = \sqrt{\text{größter EW von $A^T$A}} \ \text{(Spektralnorm)}$
	\item $\displaystyle ||A||_\infty = \max_{i=1,\ldots,n} \sum_{j=1}^n |a_{ij}| \ \text{(Zeilensummennorm)}$
\end{enumerate}
\end{flashcard}

\begin{flashcard}[Satz]{über die Cholesky-Zerlegung}
Eine invertierbare Matrix $A$ besitzt genau dann eine Cholesky-Zerlegung $A = LL^T$ mit unterer Dreiecksmatrix $L$, wenn $A$ eine spd-Matrix ist.
\end{flashcard}

\begin{flashcard}[Algorithmus]{Cholesky-Verfahren für spd-Matrizen}
Zum Lösen von
$$Ax = L \underbrace{L^T x}_{=:y} = b$$
gehen wir wie folgt vor:
\begin{enumerate}
	\item Löse durch Vorwärtssubstitution $Ly = b$.
	\item Löse durch Rückwärtssubstitution $L^Tx = y$.
\end{enumerate}
\end{flashcard}

\begin{flashcard}[Algorithmus]{Cholesky-Zerlegung}
\begin{algorithm}[H]
	\KwIn{$A$}
	\KwOut{$L$}
	Berechne $L(1,1) = \sqrt{A(1,1)}$;\\
	\For{$k = 2, \ldots, n$}{
		Löse $L(1:k-1,1:k-1)x = A(1:k-1,k)$ durch Vorwärtssubstitution;\\
		Setze $L(k,1:k-1) = x^T$;\\
		Berechne $L(k,k) = \sqrt{A(k,k) - x^Tx}$;
	}
\end{algorithm}
\end{flashcard}

\begin{flashcard}[Satz]{Gauß (Lineare Ausgleichsprobleme)}
Seien $A \in \mathbb{R}^{m \times n}, b \in \mathbb{R}^m$ mit $m > n$. Der Vektor $x \in \mathbb{R}^n$ ist genau dann eine Lösung des linearen Ausgleichsproblems $||Ax -b||_2 = min$, falls er die sogenannte Normalengleichung
$$A^TAx = A^Tb$$
löst. Insbesondere ist das lineare Ausgleichsproblem genau dann eindeutig lösbar, wenn der Rang $A$ maximal ist, d.h. $Rang(A) = n$ gilt.
\end{flashcard}