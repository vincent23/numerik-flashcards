\cardfrontfoot{Lineare Gleichungssysteme}

\begin{flashcard}[Algorithmus]{Rückwärtssubstitution}
\begin{algorithm}[H]
	\KwIn{$R, y$}
	\KwOut{$x$}
	$x_n = y_n/r_{nn};$\\
	\For{$i = n-1, \ldots, 1$}{
		$sum = r_{in}x_n;$\\
		\For{$j = i+1; \ldots, n-1$}{
			$sum = sum + r_{ij}x_j;$
			}
		$x_i = (y_i - sum) / r_{ii};$
		}
\end{algorithm}
\end{flashcard}

\begin{flashcard}[Algorithmus]{LR-Zerlegung}
\begin{enumerate}
	\item Bestimme Matrizen $P$, $L$ und $R$ mit $PA = LR$ durch Gauß-Elimination.
	\item Löse $Ly = Pb$ (Vorwärtssubstitution).
	\item Löse $Rx = y$ (Rückwärtssubstitution).
\end{enumerate}
\end{flashcard}

\begin{flashcard}[Aufwand]{Vorwärts-, Rückwärtssubstitution}
$$\sum_{i=1}^{n-1}i = n \frac{n-1}{2} \approx \frac{n^2}{2}$$
\end{flashcard}

\begin{flashcard}[Aufwand]{LR-Zerlegung}
$$\sum_{j=1}^{n-1}j^2 = \frac{(n-1)n(2n-1)}{6} \approx \frac{n^3}{3}$$
\end{flashcard}

\begin{flashcard}{Spaltenpivotwahl LR-Zerlegung}
$$|a_{jk}^{(k-1)}| = \max_{k \leq i \leq n} |a_{ik}^{(k-1)}|$$
Dies führt zu:\\
$l_{ij} \leq 1$ für alle $i, j$
\end{flashcard}