\cardfrontfoot{cg-Verfahren}

\begin{flashcard}[Definition]{Energienorm}
Ist $A \in \mathbb{R}^{n \times n}$ eine symmetrische und positiv definite Matrix, so wird durch
$$||x||_A = \sqrt{x^TAx}, \qquad x \in \mathbb{R}^n$$
eine Norm auf $\mathbb{R}^n$ definiert, die sogenannte \textbf{Energienorm}. Das entsprechende Skalarprodukt lautet
$$\langle x, y \rangle_A = x^TAy, \qquad x,y \in \mathbb{R}^n.$$
\end{flashcard}

\begin{flashcard}[Satz]{über die Konvergenz}
Nach höchstens $n$ Schritten liefert das cg-Verfahren die exakte Lösung von $Ax = b$.
\end{flashcard}

\begin{flashcard}[Algorithmus]{cg-Verfahren}
Wähle beliebiges $x^0 \in \mathbb{R}^n$ und definiere $d^0 = r^0 = b-Ax^0$ und $k=0$.\\
Solange ($||r^k|| >$ TOL $ \cdot \ ||b||$) berechne iterativ
\begin{enumerate}
	\item $\alpha_k = \langle r^k, r^k \rangle /  \langle d^k, d^k \rangle_A$,
	\item $x^{k+1} = x^k + \alpha_k d^k$,
	\item $r^{k+1} = r^k - \alpha_k Ad^k$,
	\item $\beta_k = \langle r^{k+1}, r^{k+1} \rangle / \langle r^k, r^k \rangle$,
	\item $d^{k+1} = r^{k+1} + \beta_k d^k$,
	\item Erhöhe k um 1.
\end{enumerate}
\end{flashcard}

\begin{flashcard}[Aufwand]{cg-Verfahren}
Der Aufwand wird dominiert von dem Matrix-Vektor-Produkt $Ad^k$ pro Iteration. Der Aufwand aller anderen Berechnungen einer Iteration ist linear in $n$. Bei dünn besetzten Matrizen (d.h. viele Einträge sind 0) kann das Produkt $Ad^k$ entsprechend effizienter ausgerechnet werden. Das cg-Verfahren eignet sich somit insbesondere für symmetrisch, positive definite Matrizen, die dünn besetzt sind.
\end{flashcard}

\begin{flashcard}[Satz]{Fehlerabschätzung des cg-Verfahrens (1)}
Sei $q_k$ ein beliebiges Polynom vom Grad $\leq k$ mit $q(0) = 1$, so gilt
$$||x^* - x^k||_A \leq \max_{\lambda \text{ EW von } A} |q_k(\lambda)| \cdot ||x^* - x^0||_A.$$
\end{flashcard}

\begin{flashcard}[Satz]{Fehlerabschätzung des cg-Verfahrens (2)}
Bezeichne $x^*$ die Lösung des linearen Gleichungssystems. Dann gilt:
$$||x^* - x^k||_A \leq 2 \left( \frac{\sqrt{\kappa}-1}{\sqrt{\kappa}+1}\right)^k ||x^* - x^0||_A$$
für $k = 1, 2, \ldots$ mit $\kappa = cond_2(A)$.
\end{flashcard}