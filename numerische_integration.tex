\cardfrontfoot{Numerische Integration}

\begin{flashcard}[Definition]{Rechteckregel}
Approximation durch Rechteck (Höhe am linken Rand):
$$ I(f) \approx (b - a) f(a) $$
\end{flashcard}

\begin{flashcard}[Definition]{Mittelpunktregel}
Approximation durch Rechteck (Höhe am Mittelpunkt):
$$ I(f) \approx (b - a) f\left(\frac{a + b}{2}\right) $$
\end{flashcard}

\begin{flashcard}[Definition]{Trapezregel}
Approximation durch lineare Funktion:
$$ I(f) \approx (b - a) \frac{f(a) + f(b)}{2} $$
\end{flashcard}

\begin{flashcard}[Definition]{Simpsonregel}
Lege Parabel durch drei Punkte:
$$ I(f) \approx \frac{b-a}{6} \left( f(a) + 4 f\left(\frac{a+b}{2}\right) + f(b) \right) $$
\end{flashcard}

\begin{flashcard}[Ungleichung]{Kondition der numerischen Integration}
Seien $\bar{f}, f : [a, b] \rightarrow \mathbb{R}$ stetig. Dann gilt

$$ \frac{|I(f) - I(\bar{f})|}{|I(f)|} \leq \operatorname{cond}_1 \frac{\|f-\bar{f}\|_1}{\|f\|_1}, \, \text{mit} \operatorname{cond}_1 := \frac{I(|f|)}{|I(f)|}.$$
\end{flashcard}

\begin{flashcard}[Definition]{Quadraturformel (allgemeine Formel)}
$b_i$ Gewichte, $c_i$ Knoten

$$ \int_a^b f(x) dx \approx (b - a) \sum_{i=1}^{s} b_i f(a + c_i (b - a)). $$
\end{flashcard}

\begin{flashcard}[Definition]{Ordnung einer Quadraturformel}
	Eine Quadraturformel $(b_i, c_i)_{i=1,\ldots,s}$ besitzt die \emph{Ordnung} $p$, falls sie exakte Lösungen für alle Polynome vom Grad $\leq p - 1$ liefert, wobei $p$ maximal ist.
\end{flashcard}

\begin{flashcard}[Satz]{Charakterisierung der Ordnung}
Eine Quadraturformel besitzt genau dann die Ordnung $p$, falls

$$ \frac{1}{q} = \sum_{i=1}^s b_i c_i^{q-1} \text{ für alle } q = 1, \ldots, p,$$

aber nicht mehr für $ q = p + 1$ gilt.
\end{flashcard}

\begin{flashcard}[Definition]{symmetrische Quadraturformel}
	Die Knoten sind symmetrisch zum Punkt $\frac{1}{2}$ verteilt und der Gewichtungsvektor "`rückwärts gleich wie vorwärts"', also

$$ c_i = 1 - c_{s+1-i},\qquad b_i = b_{s + 1 - i}$$

Die Ordnung einer symmetrischen Quadraturformel ist gerade.
\end{flashcard}

\begin{flashcard}[Satz]{Bedingung für erhöhte Ordnung}
	Sei $(b_i, c_i)_{i=1,\ldots,s}$ eine Quadraturformel der Ordnung $p \geq s$. Die Ordnung ist genau dann $s + m$, falls

	$$ \int_0^1 M(x) g(x) dx = 0, \text{ mit } M(x) = (x-c_1)(x-c_2) \dots (x-c_s)$$

	für alle Polynome $g$ vom Grad $\leq m - 1$, aber nicht mehr für ein Polynom vom Grad $m$ gilt.
\end{flashcard}

\begin{flashcard}[Satz]{Maximale Ordnung einer Quadraturformel}
	Die maximale Ordnung einer Quadraturformel ist $2s$.
\end{flashcard}

\begin{flashcard}[Satz]{Gauß-Quadraturformeln}
	Es existiert eine eindeutige Quadraturformel der Ordnung $2s$. Sie ist gegeben durch

	$$ c_i = \frac{1}{2}(1 + \gamma_i) $$

	für $i = 1,\ldots,s$, wobei $\gamma_i,\ldots,\gamma_s$ die Nullstellen des Legendre-Polynoms vom Grad s sind.
	Die Gewichte $b_i$ sind über die Ordnungsbedingungen eindeutig bestimmt.
\end{flashcard}
