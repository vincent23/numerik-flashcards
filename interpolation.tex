\cardfrontfoot{Interpolation}
\begin{flashcard}[Definition]{Weierstraßscher Approximationssatz}
Sei eine stetige Funktion $f: [a, b] \rightarrow \mathbb{R}$ gegeben.
Dann existiert für jedes $\epsilon > 0$ eine natürliche Zahl $n$ und ein Polynom
$p : [a, b] \rightarrow \mathbb{R}$ vom Grad $n$ mit
$$
	\max\limits_{x \in [a,b]} |f(x) - p(x)| \leq \epsilon
$$.
\end{flashcard}

\begin{flashcard}[Definition]{Lagrangesche Interpolationsformel}
Zu $n+1$ Stützpunkten $(x_i,f_i), i=0,\ldots,n$
mit paarweise verschiedenen Stützstellen $x_i$ existiert
genau ein Interpolationspolynom $p(x)$ vom Grad $\leq n$,  gegeben durch
$$
	p(x) = \sum_{i=0}^{n} f_i L_i(x)
$$
wobei $L_i(x)$ die Lagrange-Polynome der Stützstellen $x_i$ sind:
$$
	L_i(x) = \prod_{j=0, j \neq i}^{n} \frac{x-x_j}{x_i-x_j}
$$
\end{flashcard}

\begin{flashcard}[Definition]{Lebesgue-Konstante}
Wir nennen
$$
	\Lambda := \max\limits_{x \in [a,b]} \sum_{i=0}^{n} |L_i(x)|
$$
die Lebesgue-Konstante bezüglich der Stützstellen $x_0,\ldots,x_n$ auf dem Intervall $[a,b]$
\end{flashcard}

