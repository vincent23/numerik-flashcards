\cardfrontfoot{Interpolation}
\begin{flashcard}[Definition]{Weierstraßscher Approximationssatz}
Sei eine stetige Funktion $f: [a, b] \rightarrow \mathbb{R}$ gegeben.
Dann existiert für jedes $\epsilon > 0$ eine natürliche Zahl $n$ und ein Polynom
$p : [a, b] \rightarrow \mathbb{R}$ vom Grad $n$ mit
$$
	\max\limits_{x \in [a,b]} |f(x) - p(x)| \leq \epsilon
$$.
\end{flashcard}

\begin{flashcard}[Definition]{Lagrangesche Interpolationsformel}
Zu $n+1$ Stützpunkten $(x_i,f_i), i=0,\ldots,n$
mit paarweise verschiedenen Stützstellen $x_i$ existiert
genau ein Interpolationspolynom $p(x)$ vom Grad $\leq n$,  gegeben durch
$$
	p(x) = \sum_{i=0}^{n} f_i L_i(x)
$$
wobei $L_i(x)$ die Lagrange-Polynome der Stützstellen $x_i$ sind:
$$
	L_i(x) = \prod_{j=0, j \neq i}^{n} \frac{x-x_j}{x_i-x_j}
$$
\end{flashcard}

\begin{flashcard}[Bemerkung]{Lagrange-Polynome}
Interpretieren wir $\mathbb{P}_n := \lbrace p | p$ ist Polynom vom Grad $\leq n \rbrace$
als Vektorraum, so sind die Lagrange-Polynome $L_i$ bezüglich des Skalarprodukts

$$
	\langle p, q \rangle := \sum_{i=0}^{n} p(x_i) q(x_i)
$$
eine Orthonormalbasis und $p(x_i)$ eine entsprechende Linearkombination.
\end{flashcard}

\begin{flashcard}[Definition]{Lebesgue-Konstante}
Wir nennen
$$
	\Lambda := \max\limits_{x \in [a,b]} \sum_{i=0}^{n} |L_i(x)|
$$
die Lebesgue-Konstante bezüglich der Stützstellen $x_0,\ldots,x_n$ auf dem Intervall $[a,b]$
\end{flashcard}

\begin{flashcard}[Satz]{Lebesgue-Konstante}
Seien $p(x)$ und $\bar{p}(x)$ die Interpolationspolynome zu den Stützpunkten $(x_i, f_i)$
bzw. $(x_i, \bar{f}_i), i=0,\ldots,n$. Dann gilt:
$$
	\max\limits_{x \in [a,b]} |p(x) - \bar{p}(x)| \leq \Lambda_n \cdot \max\limits_{i=0,\ldots,n} |f_i - \bar{f}_i|
$$
wobei $\Lambda_n$ die kleinste Zahl mit dieser Eigenschaft ist.
\end{flashcard}

\begin{flashcard}[Satz]{Newtonsche Interpolationsformel}
Zu $n+1$ Stützpunkten $(x_i, f_i), i=0, \ldots , n$ mit paarweise verschiedenen Stützstellen $x_i$
existiert genau ein Interpolationspolynom $p(x)$ vom Grad $\leq n$, welches gegeben ist durch
$$
	p(x) = f_{0,0} + f_{0,1}(x-x_0) + \ldots + f_{0,n}(x-x_0) \cdot \ldots \cdot (x-x_{n-1}),
$$
wobei die dividierten Differenzen gegeben sind durch
$$
	f_{i,i} = f_i
	f_{i,i+k} = \frac{f_{i,i+k-1}-f_{i+1,i+k}}{x_i-x_{i+k}}
$$
für $i=0, \ldots , n$ und $1 \leq k \leq n-i$.
\end{flashcard}

\begin{flashcard}[Satz]{über den Interpolationsfehler}
Sei $f: [a,b] \rightarrow \mathbb{R}$ mindestens $(n+1)$-mal stetig differenzierbar und $p(x)$
das Interpolationspolynom von $f$ in den Stützstellen $x_0, \ldots , x_n \in [a,b]$ vom Grad $\leq n$.
Dann existiert zu jedem $x \in [a,b]$ eine Zwischenstelle $\xi = \xi(x) \in (a,b)$ mit
$$
	f(x)-p(x) = w_{n+1}(x) \cdot \frac{f^{(n+1)}(\xi)}{(n+1)!}.
$$
\end{flashcard}

\begin{flashcard}[Definition]{Tschebyscheff Polynome}
$$
	T_0(x) = 1
$$
$$
	T_1(x) = x
$$
$$
	T_{n+1}(x) = 2x \cdot T_n(x) - T_{n-1}(x).
$$
Das Polynom $T_n$ vom Grad $n$ ist ebenfalls gegeben durch:
$$
	T_n(x) = cos(n \cdot arccos x)
$$
\end{flashcard}

\begin{flashcard}[Satz]{Nullstellen Tschebyscheff Polynome}
Unter allen $(x_0, \ldots, x_n)^T \in \mathbb{R}^{n+1}$ wird $\max\limits_{x \in [-1,1]} |w_{n+1}(x)|$
minimal, wenn die $x_i$ genau die Nullstellen des $(n+1)$-ten Tschebyscheff-Polynoms $T_{n+1}$ sind, d.h. wenn
$$
	x_k = cos\left(\frac{2k+1}{2n+2} \pi\right) \text{ für } k=0, \ldots , n
$$
gilt. Der minimale Wert ist $2^{-n}$.
\end{flashcard}

\begin{flashcard}[Satz]{Lebesgue-Konstanten Tschebyscheff-Stützstellen}
Für die Lebesgue-Konstanten zu den Tschebyscheff-Stützstellen gilt
$$
	\Lambda_n \leq 3 \text{ für } n \leq 20
$$
$$
	\Lambda_n \leq 4 \text{ für } n \leq 100
$$
$$
	\Lambda_n \approx \frac{2}{\pi} log(n) \text{ für }  n \rightarrow \infty
$$
\end{flashcard}

\begin{flashcard}[Satz]{Tschebyscheff Interpolationsformel}
Zu $n+1$ Stützpunkten $(x_i, f_i), i = 0, \ldots , n$, wobei die Stützstellen genau den Nullstellen
des Tschebyscheff-Polynoms $T_{n+1}$ entsprechen, lässt sich das eindeutige Interpolationspolynom
$p(x) = p_{0,n}(x)$ vom Grad $\leq n$ darstellen durch
$$
	p(x) = \frac{1}{2}c_0 + c_1T_1(x) + \ldots + c_nT_n(x)
$$
mit
$$
	c_k = \frac{2}{n+1}\sum_{i=0}^n f_i cos \left(k \frac{2i+1}{2n+2}\pi \right)
$$
für $k=0, \ldots, n$.
\end{flashcard}

\begin{flashcard}[Satz]{Clenshaw-Algorithmus}
Sei p(x) ein Polynom mit
$$
	p(x) = \frac{1}{2}c_0 + c_1T_1(x) + \ldots + c_nT_n(x).
$$
Sei weiter $d_{n+1} = d_{n+1} = 0$ und
$$
	d_k = c_k + 2x \cdot d_{k+1} - d_{k+2} \text{ für } k = n, n-1, \ldots, 0.
$$
Dann gilt:
$$
	p(x) = \frac{1}{2}(d_0 - d_2).
$$
\end{flashcard}

\begin{flashcard}[Aufwand]{Clenshaw-Algorithmus}
Aufwand zur Berechnung von $p(x)$:
$$
	n+2 \text{ Multiplikationen}
	2n \text{ Additionen}
$$
\end{flashcard}

\begin{flashcard}[Satz]{über die Stabilität des Clenshaw-Algorithmus}
Sei $p(x)$ wie im Satz über den Clenshaw-Algorithmus definiert und $\tilde{d}_k$ durch folgende Rekursion berechnet:
$$
	0 = \tilde{d}_{n+2} = \tilde{d}_{n+1}
$$
$$
	\tilde{d}_k = c_k + 2x \cdot \tilde{d}_{k+1} - \tilde{d}_{k+2} + \epsilon_k
$$
für $k = n,n-1, \ldots , 0$, wobei $\epsilon_k$ zum Beispiel Rundungsfehler im k-ten Schritt sind. Dann gilt:
$$
	| \underbrace{\frac{1}{2}(\tilde{d}_0 - \tilde{d}_2)}_{=:\tilde{p}(x)} - p(x) | \leq \sum_{i=0}^n |\epsilon_i|
$$
für $|x| \leq 1$.
\end{flashcard}

\begin{flashcard}[Satz]{über die Min-Eigenschaft kubischer Splines}
Sei $a = x_0 < x_1 < \ldots < x_n = b$, und seien $f,s: [a,b] \rightarrow \mathbb{R}$ zweimal stetig differenzierbar mit
$$
	s(x_i) = f(x_i) \text{ für } i= 0, \ldots , n.
$$
Des Weiteren gelte \underline{eine} der Bedingungen für eingespannte, natürliche und periodische Splines.
Falls $s$ auf jedem Teilintervall $[x_{i-1}, x_i]$ ein kubisches Polynom ist, so gilt:
$$
	\int_a^b | s^{\prime\prime}(x) |^2dx \leq 	\int_a^b | f^{\prime\prime}(x) |^2dx \leq
$$
\end{flashcard}

\begin{flashcard}[Definition]{eingespannter Spline}
$$
	s^\prime(a) = f^\prime(a)
$$
$$
	s^\prime(b) = f^\prime(b)
$$
\end{flashcard}

\begin{flashcard}[Definition]{natürlicher Spline}
$$
	s^{\prime\prime}(a) = 0
$$
$$
	s^{\prime\prime}(b) = 0
$$
\end{flashcard}

\begin{flashcard}[Definition]{periodischer Spline}
$$
	s^{\prime}(a) = s^{\prime}(b)
$$
$$
	s^{\prime\prime}(a) = s^{\prime\prime}(b)
$$
$$
	f(a) = f(b)
$$
$$
	f^{\prime}(a) = f^{\prime}(b)
$$
\end{flashcard}

\begin{flashcard}[Satz]{gute Kondition der Spline-Interpolation}
Bei äquidistanter Unterteilung gilt
$$
	\Lambda_n \leq 2 \text{ für alle } n \in \mathbb{N}.
$$
\end{flashcard}

\begin{flashcard}[Satz]{Fehlerabschätzung für eingespannte kubische Splines}
Sei $f \in C^4[a,b]$, $s$ interpolierender Spline mit $s^\prime(a) = f^\prime(a)$ und $s^\prime(b) = f^\prime(b)$.
Dann gilt
$$
	|f(x) - s(x)| \leq \frac{5}{384}h^4\max_{\xi \in [a,b]} |f^{(4)}(\xi)|
$$
mit $h = max_{1,\ldots ,n} h_i$.
\end{flashcard}

\begin{flashcard}[Definition]{Bernstein-Polynome}
Das $i$-te Bernstein-Polynom vom Grad $n$ bezüglich $[0,1]$ ist
$$
	B_i^n = \binom{n}{i}(1-x)^{n-i}x^i \text{ für } i=0, \ldots , n.
$$
\end{flashcard}

\begin{flashcard}[Satz]{über elementare Eigenschaften der Bernstein-Polynome}
\begin{enumerate}[(i)]
	\item Die Bernstein-Polynome sind auf $[0,1]$ nicht negativ
	\item Symmetrie: $B_i^n = B_{n-1}^n(1-x)$ für $ i=0, \ldots , n$.
	\item Die $B_i^n$ bilden eine Partition der Eins, d.h.
	$$
		1 = \sum_{i=0}^n B_i^n(x) \text{ für } x \in \mathbb{R}.
	$$
	\item $B_i^n$ hat in $[0,1]$ genau ein Maximum bei $x = i/n$.
	\item Die $B_i^n$ bilden eine Basis von $\mathbb{P}_n$.
\end{enumerate}
\end{flashcard}

\begin{flashcard}[Satz]{rekursive Berechnung von Bernstein-Polynomen}
Für $i=1, \ldots , n-1$ und $x \in \mathbb{R}$ gilt:
$$
	B_i^n(x) = xB_{i-1}^{n-1}(x) + (1-x)B_i^{n-1}(x).
$$
\end{flashcard}

\begin{flashcard}[Definition]{Kontrollpunkte, Bézier-Polygon}
Sei ein Polynom $p \in \mathbb{P}_n^d$ in Bernstein-Darstellung
$$
	p(x) = \sum_{i=0}^n b_i B_i^n(x)
$$
gegeben. Die Koeffizienten $b_i \in \mathbb{R}^d$ heißen Kontrollpunkte von $p$,
der durch sie verlaufende Streckenzug heißt Bézier-Polygon.
\end{flashcard}

\begin{flashcard}[Lemma]{Ableitungen der Bernstein-Polynome}
$$
	\frac{d}{dx}B_i^n(x) =
	\begin{cases}
		-nB_0^{n-1}(x)&\text{ für } i=0,\\
		n(B_{i-1}^{n-1}(x) - B_{i}^{n-1}(x)) &\text{ für } i=1, \ldots , n-1 \\
		nB_{n-1}^{n-1}(x) &\text{ für } i=n.
	\end{cases}
$$
\end{flashcard}

\begin{flashcard}[Satz]{Bild von p liegt in konvexer Hülle}
Das Bild $p([0,1])$ von $p \in \mathbb{P}_n^d$ mit $p(x) = \sum_{i=0}^n b_i B_i^n(x)$
liegt in der konvexen Hülle der Kontrollpunkte $b_i$, d.h.
$$
	p(x) \in conv\lbrace b_0, \ldots , b_n \rbrace \text{ für } x \in [0,1].
$$
\end{flashcard}

\begin{flashcard}[Definition]{Teilpolynome}
Für $p \in \mathbb{P}_n^d$ mit $p(x) = \sum_{i=0}^n b_i B_i^n(x)$ sind durch
$$
	\beta_i^k(x) = \sum_{j=0}^k b_{i+j}B_j^k(x) \quad , k=0, \ldots , n, i=0, \ldots, n-k,
$$
die sogenannten Teilpolynome $\beta_i^k(x) \in \mathbb{P}_n^d$ definiert, wobei $k$ den Grad bezeichnet.
\end{flashcard}

\begin{flashcard}[Satz]{rekursive Berechnung der Teilpolynome}
Die Teilpolynome $\beta_i^k(x)$ von $p(x) = \sum_{i=0}^n b_i B_i^n(x)$ genügen der Rekursion
$$
	\beta_i^k(x) = (1-x)\beta_i^{k-1}(x) + x\beta_{i+1}^{k-1}(x)
$$
für $k=0, \ldots , n$ und $i = 0, \ldots , n-k$. \linebreak
Die Auswertung eines Polynoms durch die rekursive Berechnung (Algorithmus von de Casteljau)
ist für $x \in  [0,1]$ äußerst stabil, da nur Konvexkombinationen berechnet werden und somit Störungen gedämpft werden.
\end{flashcard}

\begin{flashcard}[Definition]{Bernstein-Polynome bezüglich beliebiger Intervalle}
Das $i$-te Bernstein-Polynom vom Grad $n$ bezüglich $[a,b]$ lautet
$$
	B_i^n(x; a, b) = \frac{1}{(b-a)^n}\binom{n}{i}(b-x)^{n-i}(x-a)^i
$$
für $i=0, \ldots , n$.
\end{flashcard}

\begin{flashcard}[Satz]{über Bernstein-Polynome bezüglich beliebiger Intervalle}
Sei die Bernstein-Darstellung $p(x) = \sum_{i=0}^n b_i B_i^n(x; a, b)$ zum Intervall $[a,b]$ gegeben.
Dann lassen sich die Darstellungen
$$
	p(x) = \sum_{i=0}^n a_i B_i^n(x; a, c) = \sum_{i=0}^n c_i B_i^n(x; c, b)
$$
bezüglich $[a, c]$ und $[c, b]$ für $c \in ]a, b[$ leicht berechnen durch
$$
	a_k = \beta_0^k(c; a,b), \quad c_k = \beta_k^{n-k}(c; a, b)
$$
für $k=0, \ldots ,n$.
\end{flashcard}
