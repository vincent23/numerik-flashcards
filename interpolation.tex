\cardfrontfoot{Interpolation}
\begin{flashcard}[Definition]{Weierstraßscher Approximationssatz}
Sei eine stetige Funktion $f: [a, b] \rightarrow \mathbb{R}$ gegeben.
Dann existiert für jedes $\epsilon > 0$ eine natürliche Zahl $n$ und ein Polynom
$p : [a, b] \rightarrow \mathbb{R}$ vom Grad $n$ mit
$$
	\max\limits_{x \in [a,b]} |f(x) - p(x)| \leq \epsilon
$$.
\end{flashcard}

\begin{flashcard}[Definition]{Lagrangesche Interpolationsformel}
Zu $n+1$ Stützpunkten $(x_i,f_i), i=0,\ldots,n$
mit paarweise verschiedenen Stützstellen $x_i$ existiert
genau ein Interpolationspolynom $p(x)$ vom Grad $\leq n$,  gegeben durch
$$
	p(x) = \sum_{i=0}^{n} f_i L_i(x)
$$
wobei $L_i(x)$ die Lagrange-Polynome der Stützstellen $x_i$ sind:
$$
	L_i(x) = \prod_{j=0, j \neq i}^{n} \frac{x-x_j}{x_i-x_j}
$$
\end{flashcard}

\begin{flashcard}[Bemerkung]{Lagrange-Polynome}
Interpretieren wir $\mathbb{P}_n := \lbrace p | p$ ist Polynom vom Grad $\leq n \rbrace$
als Vektorraum, so sind die Lagrange-Polynome $L_i$ bezüglich des Skalarprodukts

$$
	\langle p, q \rangle := \sum_{i=0}^{n} p(x_i) q(x_i)
$$
eine Orthonormalbasis und $p(x_i)$ eine entsprechende Linearkombination.
\end{flashcard}

\begin{flashcard}[Definition]{Lebesgue-Konstante}
Wir nennen
$$
	\Lambda := \max\limits_{x \in [a,b]} \sum_{i=0}^{n} |L_i(x)|
$$
die Lebesgue-Konstante bezüglich der Stützstellen $x_0,\ldots,x_n$ auf dem Intervall $[a,b]$
\end{flashcard}

\begin{flashcard}[Satz]{Lebesgue-Konstante}
Seien $p(x)$ und $\bar{p}(x)$ die Interpolationspolynome zu den Stützpunkten $(x_i, f_i)$
bzw. $(x_i, \bar{f}_i), i=0,\ldots,n$. Dann gilt:
$$
	\max\limits_{x \in [a,b]} |p(x) - \bar{p}(x)| \leq \Lambda_n \cdot \max\limits_{i=0,\ldots,n} |f_i - \bar{f}_i|
$$
wobei $\Lambda_n$ die kleinste Zahl mit dieser Eigenschaft ist.
\end{flashcard}

\begin{flashcard}[Satz]{Newtonsche Interpolationsformel}
Zu $n+1$ Stützpunkten $(x_i, f_i), i=0, \ldots , n$ mit paarweise verschiedenen Stützstellen $x_i$
existiert genau ein Interpolationspolynom $p(x)$ vom Grad $\leq n$, welches gegeben ist durch
$$
	p(x) = f_{0,0} + f_{0,1}(x-x_0) + \ldots + f_{0,n}(x-x_0) \cdot \ldots \cdot (x-x_{n-1}),
$$
wobei die dividierten Differenzen gegeben sind durch
$$
	f_{i,i} = f_i
	f_{i,i+k} = \frac{f_{i,i+k-1}-f_{i+1,i+k}}{x_i-x_{i+k}}
$$
für $i=0, \ldots , n$ und $1 \leq k \leq n-i$.
\end{flashcard}

\begin{flashcard}[Satz]{über den Interpolationsfehler}
Sei $f: [a,b] \rightarrow \mathbb{R}$ mindestens $(n+1)$-mal stetig differenzierbar und $p(x)$
das Interpolationspolynom von $f$ in den Stützstellen $x_0, \ldots , x_n \in [a,b]$ vom Grad $\leq n$.
Dann existiert zu jedem $x \in [a,b]$ eine Zwischenstelle $\xi = \xi(x) \in (a,b)$ mit
$$
	f(x)-p(x) = w_{n+1}(x) \cdot \frac{f^{(n+1)}(\xi)}{(n+1)!}.
$$
\end{flashcard}

\begin{flashcard}[Definition]{Tschebyscheff Polynome}
$$
	T_0(x) = 1
$$
$$
	T_1(x) = x
$$
$$
	T_{n+1}(x) = 2x \cdot T_n(x) - T_{n-1}(x).
$$
Das Polynom $T_n$ vom Grad $n$ ist ebenfalls gegeben durch:
$$
	T_n(x) = cos(n \cdot arccos x)
$$
\end{flashcard}

\begin{flashcard}[Satz]{Nullstellen Tschebyscheff Polynome}
Unter allen $(x_0, \ldots, x_n)^T \in \mathbb{R}^{n+1}$ wird $\max\limits_{x \in [-1,1]} |w_{n+1}(x)|$
minimal, wenn die $x_i$ genau die Nullstellen des $(n+1)$-ten Tschebyscheff-Polynoms $T_{n+1}$ sind, d.h. wenn
$$
	x_k = cos\left(\frac{2k+1}{2n+2} \pi\right) \text{ für } k=0, \ldots , n
$$
gilt. Der minimale Wert ist $2^{-n}$.
\end{flashcard}

\begin{flashcard}[Satz]{Lebesgue-Konstanten Tschebyscheff-Stützstellen}
Für die Lebesgue-Konstanten zu den Tschebyscheff-Stützstellen gilt
$$
	\Lambda_n \leq 3 \text{ für } n \leq 20
$$
$$
	\Lambda_n \leq 4 \text{ für } n \leq 100
$$
$$
	\Lambda_n \approx \frac{2}{\pi} log(n) \text{ für }  n \rightarrow \infty
$$

\end{flashcard}
